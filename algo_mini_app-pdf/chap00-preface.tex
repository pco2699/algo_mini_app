\chapter{まえがき}
\label{chap:chap00-preface}

\section*{自己紹介}
\addcontentsline{toc}{section}{自己紹介}
\label{sec:-1}

はじめまして、こんにちは、たかやま(@pco2699)と申します。
普段は、資産運用のスタートアップでJavaのバックエンドエンジニアとして働いています。

土日は、ハッカソンに出たり、このようなちょっとしたアプリをつくるような開発本を書いたり
プチ0{-}\textgreater{}1を楽しむのが、日課です。

\section*{Webエンジニアにアルゴリズム/データ構造の知識は必要か}
\addcontentsline{toc}{section}{Webエンジニアにアルゴリズム/データ構造の知識は必要か}
\label{sec:-2}

よくはてぶのホッテントリやTwitterでも話題に上がる、\textbf{Webエンジニアにアルゴリズム・データ構造などの知識は必要か?}論。
皆さんも目にしたことがあると思います。この論について、どのような意見をお持ちでしょうか?

私、個人としては、\textbf{無くても困らないけど、あると色々助かる。}ものだと思っています。
こういったCS(コンピュータ・サイエンス)と呼ばれる以下のような知識は、エンジニアにとっては\textbf{筋肉}のようなものだと思います。

\begin{starteritemize}
\item アルゴリズム
\item データ構造
\item ネットワーク
\item オペレーティング・システム
\end{starteritemize}

などなど...

筋肉ももちろん無くて困ることはありませんが、あるといろいろ便利です。

\section*{でも、アルゴリズムの勉強って何に使うの??}
\addcontentsline{toc}{section}{でも、アルゴリズムの勉強って何に使うの??}
\label{sec:-3}

Webエンジニアとして働いていると、もちろんアルゴリズムがそのまま出てくることなんて、そうそう無いです。
普通にWebページを作る、ということであれば、アルゴリズムの知識なんて全然使わないと思います。

しかも、町の本屋さんやAmazonで売っているいわゆる「アルゴリズム・データ構造」が載っている本は
ストイックにアルゴリズムやデータ構造が書いてあるので、まあ読む気が失せるわけです。

そこで、私はアルゴリズムをそのまま小さいWebアプリ = アルゴリズムミニアプリ として
実装することで、「アルゴリズムがWebアプリでどのように活かされるか」をわかりやすく理解できる本を書きたいと思いました。

それが本書「Nuxt.jsとPythonでつくるアルゴリズムミニアプリ」です。

\section*{アルゴリズムミニアプリってなに?}
\addcontentsline{toc}{section}{アルゴリズムミニアプリってなに?}
\label{sec:-4}

「アルゴリズムミニアプリってなんだろう」と思う方もいると思います。
アルゴリズムミニアプリは、以下の要素をもつWebアプリです。(この概念自体はもちろん、自分が作ったものです。)

\begin{starteritemize}
\item Facebookの友達検索機能!など、Webサービスの一つの機能だけをアプリとして切り出したもの。
\item ゲーム好きの方なら「メイドインワリオ」のWebサービス 
\end{starteritemize}

\section*{対象読者}
\addcontentsline{toc}{section}{対象読者}
\label{sec:-5}

本書の対象読者は以下のような方です。
特に1.の方を対象としています。

\begin{starterenumerate}
\item アルゴリズムがよくわからないWeb系のエンジニア
\item アルゴリズムがどのようにWebアプリに活用されるか知りたい人
\item CS専攻で、モダンなWeb技術でのアプリの技術スタック、開発方法が知りたい人
\end{starterenumerate}

言語はJavaScriptをメインで利用します。
JavaScriptを書いたことが無くても、C, C++, Java, Python, Ruby, PHPなどの言語をどれか一つで
簡単なWebシステムやツールを書いたことがある方なら理解できる内容となっていると思います。

逆に以下のような方は、本書を読んでも??となってしまうかもしれません。

\begin{starterenumerate}
\item プログラミングを全くしたことが無い方
\end{starterenumerate}

\section*{本書の構成}
\addcontentsline{toc}{section}{本書の構成}
\label{sec:-6}

本書は、各章で一つのアルゴリズムを取り上げます。
章の前半でアルゴリズムの解説、後半でWebアプリの説明、およびコード内容の説明を行います。
